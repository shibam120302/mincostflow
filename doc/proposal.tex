\section{Proposal}

% Based on the description of the problem in the introduction
This Summer of Bitcoin project, proposes the design and implementation of a C++
mini-library with no external dependencies for solving the min cost flow problem
(MCF) with convex costs that can be used to optimize the probability of success
of multipart payments in the Lightning Network.

Some time of the project will be allocated to study of the state of the art of
the problem. At the present we are aware of two algorithms that can be used to
solve the MCF with convex costs.
One that transforms approximates the cost function to a piecewise linear
function and splits the arc into several linear cost arcs accordingly (section
14.4 of \cite{ahuja1993network}).
And the so called \emph{Capacity}
algorithm described in section 14.5 of \cite{ahuja1993network},
which is a generalization of a \emph{Excess Scaling} solver for the linear
MCF problem.
During the course of this project duration, we will evaluate which algorithm
will be best suited for the problem domain.

Parallelization of the algorithms will be considered and discussed during the
development of the back-end.
We will propose to use standard library tools for that purpose
which are available in C++17.

For the preparation of this Summer of Bitcoin project proposal, we have been
working on a proof-of-concept MCF C++ library\footnote{%
\url{https://github.com/Lagrang3/mincostflow}} publicly available on Github.
The API of this MCF library permits the representation of a directed graph
\texttt{digraph} in a space efficient data structure.
It also allows for the computation 
shortest paths on a weighted network using Dijsktra's and Breadth-First-Search
algorithms, which serve as building blocks for flow solvers.
There are two different back-ends for the solution of the Max Flow problem:
using Edmond-Karp's \emph{Augmenting Paths} and Dinic's \emph{Push-Relabel}.
Also these constitute building blocks for the Min Cost Flow solvers.
The library contains a MCF solver based on the \emph{Successive Shortest Path}
algorithm presented in section 9.7 on \cite{ahuja1993network}.

In the library's respository we provide examples of applications that we have also
used as test for correctedness and efficiency of our implementation.
These examples actually solve two problems from the online judge Kattis:
\texttt{maxflow}\footnote{\url{https://open.kattis.com/problems/maxflow}}
and
\texttt{mincostmaxflow}\footnote{\url{https://open.kattis.com/problems/mincostmaxflow}},
and they do pass the test cases within the time and memory constraints.
